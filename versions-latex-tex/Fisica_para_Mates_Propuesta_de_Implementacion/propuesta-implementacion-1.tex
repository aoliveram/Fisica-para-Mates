\documentclass[12pt, a4paper]{report}
\usepackage[spanish]{babel}
\usepackage[utf8]{inputenc}
\usepackage{amsmath, amssymb, amsfonts}
\usepackage{array}
\usepackage{float} % For H option in figure/table
\usepackage{siunitx} % For units
\usepackage{geometry}
\usepackage{graphicx}
\usepackage{soul}
\usepackage{rotating} % Para texto vertical en cabeceras de tabla
\usepackage{multicol} % Para la leyenda
\usepackage{booktabs} % Para líneas más estéticas en la tabla
\usepackage{hyperref}
\usepackage{xcolor}
% Compilar con: lualatex -shell-escape o xelatex -shell-escape
\usepackage{listings}
\usepackage{tcolorbox}
\newtcolorbox{solutionbox}{
    colback=gray!5!white,
    colframe=gray!75!black,
    title=\textbf{Solución esperada},
    fonttitle=\bfseries,
    coltitle=white,
    arc=2mm,
    boxrule=0.5mm
}


\lstdefinestyle{mypython}{
  language=Python,
  basicstyle=\ttfamily\footnotesize,
  numbers=left, numbersep=6pt,
  breaklines=true,
  showstringspaces=false,
  tabsize=4,
  keywordstyle=\bfseries\color{blue!60!black},
  stringstyle=\color{green!40!black},
  commentstyle=\itshape\color{black!50},
  frame=single,
  rulecolor=\color{black!20}
}

\geometry{a4paper, margin=2cm} % Ajuste para tabla ancha y modo paisaje

\setcounter{tocdepth}{1} % Define la profundidad del índice
\renewcommand{\familydefault}{\sfdefault} % Sans-serif

% Comandos para texto vertical en cabeceras de tabla
\newcommand{\rockbottom}[1]{\rotatebox{90}{\parbox{2.8cm}{\raggedright #1}}}
\newcommand{\rockbottomshort}[1]{\rotatebox{90}{\parbox{2.2cm}{\raggedright #1}}}
\newcommand{\rockbottomtiny}[1]{\rotatebox{90}{\parbox{1.8cm}{\raggedright #1}}}
\newcommand{\rockbottomsupertiny}[1]{\rotatebox{90}{\parbox{1.5cm}{\raggedright #1}}}

\geometry{margin=1in}

\title{\textbf{Física para Mates \\
Implementación Piloto}}
\author{Aníbal Olivera M.}
\date{Noviembre 2025}

\begin{document}

\maketitle

\tableofcontents

\newpage

%======================================================================
\chapter{Contexto y objetivos generales}
%======================================================================

\section{Contexto}

La presente propuesta describe una implementación piloto factible del proyecto \textit{Física para Mates} en los cursos de Geometría y Taller de Física I de la carrera Ingeniería Civil Plan Común de la Facultad de Ingeniería de la Universidad del Desarrollo. La presente implementación:

\begin{itemize}
\item utiliza problemas ya diseñados y acompañados de simulaciones computacionales;
\item se integra a la estructura de evaluación existente sin modificar sus ponderaciones;
\item incorpora mediciones pre–post de motivación y estrategias de aprendizaje;
\item requiere infraestructura ya disponible en la universidad (salas de computación) y un esfuerzo acotado de coordinación entre las cátedras de Matemática y Física.
\end{itemize}

Este documento está pensado para ser leído por académicos y autoridades que no han participado en la discusión previa, por lo que se explicitan los objetivos, la estructura de las sesiones, la articulación con la malla y el sistema de evaluación.

En una primera etapa, el proyecto se ha concretado en el diseño de una serie de problemas aplicados acompañados de simulaciones computacionales interactivas. Estos problemas ya se encuentran detallados en el documento \textit{Física\_para\_Mates\_AO\_v2.pdf}, y tienen en común 1) hacer uso de conceptos de Física mínimos —cinemática y ecuación de la trayectoria— y 2) ser no triviales en su desarrollo, enfatizando el rol de las simulaciones como 'bombas de intuición'.

En este informe se propone una implementación piloto de dichos problemas en dos asignaturas del Plan Común: \textbf{Geometría} (bimestre 2, con clases diarias de 140 minutos) y \textbf{Taller de Física I} (semestre 2, con 3 clases semanales de 70 minutos más una ayudantía).

Dependiendo de la evaluación de esta primera experiencia, el proyecto puede escalarse en cursos posteriores (por ejemplo, Introducción al Cálculo y Cálculo Diferencial), incorporando nuevos problemas y ampliando el uso de simulaciones en la formación básica de los estudiantes de Ingeniería.

\section{Objetivos generales del proyecto}

\begin{enumerate}
  \item Conectar explícitamente contenidos de Matemática (ecuaciones de segundo grado, funciones reales, geometría analítica, trigonometría) con escenarios de problemas de Física estándar, aunque revisitados y visto desde nuevas ópticas.
  \item Integrar \textbf{simulaciones computacionales} como herramienta de exploración (bombas de intuición), modelación y verificación de resultados matemáticos.
  \item Desarrollar habilidades de resolución de \textbf{problemas no triviales}, donde los estudiantes deban interpretar, modelar y analizar situaciones con parámetros múltiples.
  \item \textbf{Evaluar el impacto} de esta intervención en variables de motivación y estrategias de aprendizaje mediante un cuestionario estandarizado aplicado en modalidad pre--post.
\end{enumerate}

\section{Objetivos específicos del proyecto}

\begin{itemize}
  \item Diseñar e implementar un conjunto acotado de sesiones de trabajo en grupos de tres estudiantes, utilizando simulaciones computacionales en salas de computadores de la universidad.
  \item Integrar estas sesiones dentro de la estructura de evaluación de los cursos de Geometría y Taller de Física I, sin alterar sus ponderaciones globales.
  \item Recoger datos de desempeño (productos escritos, mini--evaluaciones) y datos de percepción y motivación mediante un cuestionario tipo \textbf{MSLQ} (Motivated Strategies for Learning Questionnaire) adaptado a estudiantes de ingeniería de primer año.
\end{itemize}

\section{Resumen de problemas disponibles}

\vspace{0.5cm}

\begin{table}[H]
\centering
\footnotesize
\setlength{\tabcolsep}{3pt}
\begin{tabular}{@{}lcccccccccccc@{}}
\toprule
 & \multicolumn{5}{c}{\textbf{Contenidos Matemáticos}} & \multicolumn{3}{c}{\textbf{Contenidos Físicos}} & \multicolumn{3}{c}{\textbf{Tipo de Problema}} & \multicolumn{1}{c}{\textbf{Asignatura}} \\
\cmidrule(lr){2-6} \cmidrule(lr){7-9} \cmidrule(lr){10-12} \cmidrule(lr){13-13}
\textbf{Problema} &
\rockbottomtiny{Trigonometría} &
\rockbottomtiny{Vectores} &
\rockbottomtiny{Geom. Analítica} &
\rockbottomtiny{Ecuac. de segundo gr.} &
\rockbottomtiny{Cálc. Diferen.} &
\rockbottomtiny{Cinemática} &
\rockbottomtiny{Leyes Newton} &
\rockbottomtiny{Estática /Equilibrio} &
\rockbottomsupertiny{Hoja Cálculo} &
\rockbottomsupertiny{Prob. Pizarra} &
\rockbottomsupertiny{Simul. 3D} &
\rockbottomshort{Sugerida} \\
\midrule
Cinemática 1 (4.1)   &   &   &   & X &   & X &   &   &   & X & X & 1 \\
Balística 1 (4.2)    & X & X & X &   &   & X &   &   &   & X & X & 1 \\
Balística 2 (4.3)    & X & X & X &   &   & X &   &   &   & X & X & 1 \\
Balística 3 (4.4)    &   & X & X &   &   & X &   &   &   & X & X & 1 \\
Sólido Rev. 1 (4.5)  & X & X & X &   &   & X & X & X &   & X & X & 1 \\
\midrule
Sólido Rev. 2 (5.1)  &   & X & X &   & X & X & X & X &   &   & X & 3 \\
\bottomrule
\end{tabular}
\end{table}

\vspace{0.3cm}
\noindent\textbf{Asignatura sugerida:}
\begin{itemize}
  \item[1:] Geometría
  \item[2:] Taller de Física I (en articulación con Introducción al Cálculo)
  \item[3:] Cálculo Diferencial
  \item[4:] Cálculo Integral
\end{itemize}

En este documento nos centraremos en los problemas 4.1--4.5 y 5.1, articulados entre Geometría y Taller de Física I.

%======================================================================
\chapter{Diseño general de la implementación}
%======================================================================

\section{Cursos involucrados y carga horaria}

\subsection*{Geometría}

\begin{itemize}
  \item Curso del de primer año, bimestral, entre mediados de mayo y principio de julio.
  \item Clases de 140 minutos, de lunes a viernes, durante 8 semanas.
  \item Para esta propuesta se utilizarán sesiones de 140 minutos, agendadas en las \textbf{últimas semanas} del bimestre.
\end{itemize}

\subsection*{Taller de Física I (por confirmar*)}

\begin{itemize}
  \item Curso de primer año, articulado con Introducción al Cálculo.
  \item Clases de 70 minutos, 3 veces por semana + 1 ayudantía.
  \item Para esta propuesta se utilizarán sesiones de 140 minutos, agendadas en las \textbf{primeras semanas} del semestre en que se dicta el Taller (inmediatamente después de Geometría).
\end{itemize}

\subsection*{Estructura G / F}

Definiremos como sesiones de \textbf{140 minutos obligatorias}:

\begin{itemize}
  \item \textbf{Gk}: sesiones realizadas en \textbf{Geometría} (ej.: G1, G2, G3, G4).
  \item \textbf{Fk}: sesiones realizadas en \textbf{Taller de Física I} (ej.: F1, F2, F3).
\end{itemize}

%–––––––––––––––––––––––––––––––––––
\section{Esquema versiones de implementación}
%–––––––––––––––––––––––––––––––––––

Con el propósito de aumentar la cartera de posibilidades para ejecutar el presente proyecto, se proponen dos versiones de implementación con distinta intensidad, ambas utilizando la misma infraestructura y estructura básica de sesión. Esto es un esquema de ambas versiones —versión mínima y versión completa.

\subsection{Versión mínima (4 sesiones)}

Pensada como piloto de baja carga, con dos sesiones en Geometría y una en Física.

\begin{itemize}
\item G1:
\begin{itemize}
\item Aplicación de cuestionario de motivación y estrategias de estudio \textbf{(pre–test)}.
\item Problema \textbf{Cinemática 1: frenada de emergencia}.
\item Mini–evaluación individual.
\end{itemize}
\item G2:
\begin{itemize}
\item Problema \textbf{Balística 2: el desafío del arquero}.
\item Mini–evaluación individual.
\end{itemize}
\item G3:
\begin{itemize}
\item Problema \textbf{Balística 3: la bola en la escalera}.
\item Mini–evaluación individual.
\end{itemize}
\item F1:
\begin{itemize}
\item Breve repaso de proyectiles y conexión con las sesiones de Geometría.
\item Problema \textbf{Sólido de revolución 1 (cono)}.
\item Mini–evaluación individual.
\item Aplicación de cuestionario \textbf{(post–test)}.
\end{itemize}
\end{itemize}

\subsection{Versión completa (7 sesiones)}

Versión extendida, con todos los problemas disponibles y una sesión de cierre.

\begin{itemize}
\item G1:
\begin{itemize}
\item Cuestionario pre–test.
\item \textbf{Cinemática 1}.
\item Mini–evaluación individual.
\end{itemize}
\item G2:
\begin{itemize}
\item \textbf{Balística 1: rescate de la pelota}.
\item Mini–evaluación individual.
\end{itemize}
\item G3:
\begin{itemize}
\item \textbf{Balística 2: desafío del arquero}.
\item Mini–evaluación individual.
\end{itemize}
\item G4:
\begin{itemize}
\item \textbf{Balística 3: bola en la escalera}.
\item Mini–evaluación individual.
\end{itemize}
\item F1:
\begin{itemize}
\item Conexión entre proyectiles y movimiento circular.
\item \textbf{Sólido de revolución 1 (cono)}.
\item Mini–evaluación individual.
\end{itemize}
\item F2:
\begin{itemize}
\item \textbf{Sólido de revolución 2}: superficies de revolución (paraboloide, esfera, hiperbola, elipse).
\item Mini–evaluación individual.
\end{itemize}
\item F3:
\begin{itemize}
\item Cierre del proyecto: revisión integradora de todos los problemas.
\item Mini–evaluación integradora.
\item Cuestionario post–test.
\end{itemize}
\end{itemize}

%======================================================================
\chapter{Estructura de una sesión tipo}
%======================================================================

Todas las sesiones especiales se imparten como \textbf{clases dobles de 140 minutos}, con estudiantes trabajando en grupos de tres, cada grupo con acceso a un computador con conexión a las simulaciones.

A continuación se describen las fases de una sesión tipo Gk/Fk (exceptuando las partes de pre/post–test, que se detallan más adelante):

\section{Fases de la sesión}

\begin{enumerate}
\item \textbf{Introducción y contexto (15–20 min)}
El profesor presenta brevemente el contexto físico y el objetivo matemático del problema. Se explican las reglas de trabajo en grupo y los productos que se esperan: \textit{hoja de desarrollo} y mini–evaluación individual.

\item \textbf{Exploración inicial con simulación (10–15 min)}
Los estudiantes exploran libremente la simulación asociada al problema, modificando parámetros y observando comportamientos cualitativos. El objetivo es familiarizarse con el sistema y generar hipótesis (por ejemplo, parámetros para evitar una colisión, combinaciones de ángulo y velocidad que permiten pasar por dos anillos, etc.).

\item \textbf{Desarrollo de actividades (50–65 min)}
Esta es la fase central de la sesión. Los grupos desarrollan, sobre la base de una guía impresa o digital, una serie de actividades que combinan:
\begin{itemize}
\item formulación de ecuaciones (de posición, trayectorias, condiciones de impacto);
\item manipulación algebraica y trigonométrica;
\item resolución de inecuaciones y análisis de discriminantes;
\item comparación de resultados analíticos con la simulación.
\end{itemize}
Las respuestas y procedimientos se registran en la \textbf{hoja de desarrollo grupal}, que se entrega al final de la sesión.

\item \textbf{Verificación y cierre grupal (10–15 min)}
El profesor conduce una discusión rápida donde se contrastan los resultados de los grupos con las predicciones teóricas y con la simulación. Esta fase suele corresponder a los numerales finales de cada guía (interpretación de la condición de seguridad en cinemática, descripción de la región de soluciones en balística, análisis de cómo cambia el peldaño de impacto con la velocidad inicial, comparación entre distintas superficies de revolución, etc.).

\item \textbf{Mini–evaluación individual (15–20 min)}
Cada estudiante responde un breve cuestionario individual (3–5 ítems) relacionado con el problema trabajado. La mini–evaluación puede incluir:
\begin{itemize}
\item un ejercicio numérico directo;
\item una pregunta conceptual sobre el modelo;
\item una pregunta de interpretación gráfica o de parámetros.
\end{itemize}
Esta mini–evaluación se corrige individualmente y forma parte de la nota de la sesión.
\end{enumerate}

\section{Distribución de tiempos}

En términos de minutos, una sesión tipo puede ser esquematizada siguendo esta distribución de tiempos:

\begin{center}
\begin{tabular}{@{}lc@{}}
\toprule
\textbf{Fase} & \textbf{Tiempo estimado} \\
\midrule
Introducción y contexto & 15–20 min \\
Exploración con simulación & 10–15 min \\
Desarrollo de actividades & 50–65 min \\
Verificación y cierre grupal & 10–15 min \\
Mini–evaluación individual & 15–20 min \\
\midrule
\textbf{Total} & \textbf{100 - 135 min} \\
\bottomrule
\end{tabular}
\end{center}
\vspace{0.35cm}

El profesor a cargo debe cronometrar cada etapa y ser riguroso con los minutos asignados, toda sesión debe terminar con la entrega de la mini-evaluación individual.

%======================================================================
\chapter{Evaluación y articulación con la nota de curso}
%======================================================================

\section{Productos evaluados en cada sesión}

Cada sesión Gk o Fk genera dos productos evaluados (pueden ver la hoja de ejercicios que verán los alumnos en en archivo \texttt{Fisica\_para\_Mates\_AO\_v2-alumnos.pdf}, además de ver el banco de ejercicios para la mini-evaluación en la sección \ref{Banco de Ítems}):

\begin{enumerate}
\item \textbf{Hoja de desarrollo grupal (50\% de la sesión)} 
\begin{itemize}
\item Elaborada en grupos de tres estudiantes.
\item Contiene el desarrollo completo de las actividades de la guía correspondiente.
\item Se evalúa mediante una rúbrica simple (corrección matemática, claridad de procedimientos, interpretación de resultados).
\end{itemize}

\item \textbf{Mini–evaluación individual (50\% de la sesión)}
\begin{itemize}
\item Aproximadamente 5 ítems cortos relacionados a los ejercicios de la sesión.
\item Garantiza responsabilidad individual dentro del trabajo grupal.
\end{itemize}
\end{enumerate}

La nota de la sesión es el promedio de ambos entregables.

\section{Integración en Geometría}

La ponderación oficial de Geometría contempla:

\begin{center}
\begin{tabular}{@{}lc@{}}
\toprule
\textbf{Prueba} & \textbf{Ponderación} \\
\midrule
Certamen N$^\circ$1 & 35\% \\
Certamen N$^\circ$2 & 35\% \\
Controles & 20\% \\
Tareas & 10\% \\
\midrule
Nota de presentación & 70\% \\
Examen & 30\% \\
\bottomrule
\end{tabular}
\end{center}

En esta propuesta, la nota de \textbf{Tareas (10\%)} se reemplaza por el promedio de las notas de sesiones Gk. No se modifica la ponderación global del curso, sólo se redefine el contenido de la categoría ``Tareas’’.

\section{Integración en Taller de Física I}

La ponderación oficial de Taller de Física I contempla:

\begin{center}
\begin{tabular}{@{}lc@{}}
\toprule
\textbf{Prueba} & \textbf{Ponderación} \\
\midrule
Certamen N$^\circ$1 & 30\% \\
Certamen N$^\circ$2 & 30\% \\
Controles & 25\% \\
Laboratorios & 15\% \\
\midrule
Nota de presentación & 70\% \\
Examen & 30\% \\
\bottomrule
\end{tabular}
\end{center}

En esta propuesta, el proyecto \textit{Física para Mates} se integra dentro de la categoría \textbf{Laboratorios (15\%)}, donde cada sesión Fk reemplaza 5 puntos de esa categoría. En este sentido, la sección Laboratorio estaría compuesto por:
\begin{itemize}
    \item F1 (5\%) y Laboratorio usual (10\%).
    \item F1-F2-F3 (15\%) y Laboratorio usual (0\%).
\end{itemize}

%======================================================================
\chapter{Medición de motivación y estrategias de aprendizaje}
%======================================================================

\section{Instrumento}

Para medir el impacto de la intervención en variables de motivación y estrategias de aprendizaje, se utilizará una versión adaptada del cuestionario \textbf{MSLQ}, validado en estudiantes universitarios chilenos de primer año.

Este cuestionario evalúa dimensiones como:

\begin{itemize}
\item Valoración de la tarea.
\item Expectativas de logro y autoeficacia.
\item Ansiedad ante la evaluación.
\item Organización y planificación.
\item Estrategias de aprendizaje (resumen, elaboración, pensamiento crítico).
\item Manejo del tiempo y ambiente de estudio.
\item Búsqueda de ayuda.
\end{itemize}

En su versión completa, el instrumento contiene alrededor de 80 ítems tipo Likert. Para efectos de esta implementación piloto se sugiere considerar:

\begin{itemize}
\item Usar el cuestionario completo, destinando aproximadamente \textbf{15–20 minutos} de clase.
\item Alternativamente, diseñar una \textbf{versión abreviada} centrada en los factores más relevantes (por ejemplo, autoeficacia, valor de la tarea y ansiedad), reduciendo el tiempo de aplicación.
\end{itemize}

\section{Aplicación pre–post}

\subsection*{Versión mínima}

\begin{itemize}
\item \textbf{Pre–test}: al inicio de la sesión G1 (primeros 15–20 minutos), o durante la clase anterior.
\item \textbf{Consentimiento}: al inicio de la sesión G1, se les presentará consentimiento informado dando a conocer los alcances de la actividad (primeros 15–20 minutos).
\item \textbf{Post–test}: al final de la sesión F1 (últimos 15–20 minutos).
\end{itemize}

\subsection*{Versión completa}

\begin{itemize}
\item \textbf{Pre–test}: al inicio de la sesión G1 (primeros 15–20 minutos), o durante la clase anterior.
\item \textbf{Consentimiento}: al inicio de la sesión G1, se les presentará consentimiento informado dando a conocer los alcances de la actividad (primeros 15–20 minutos).
\item \textbf{Post–test}: durante la sesión F3 (parte inicial o final), en conjunto con la mini–evaluación integradora.
\end{itemize}

Los resultados del cuestionario no afectan la nota de los estudiantes en ningún caso. Esta información se utilizará para monitorear cambios en la \textbf{motivación} y \textbf{estrategias de estudio} frente al uso de simulaciones y problemas contextualizados; y alimentar investigaciones posteriores sobre enseñanza de Matemática y Física en Ingeniería.

%======================================================================
\chapter{Requerimientos logísticos}
%======================================================================

\section{Infraestructura}

\begin{itemize}
\item \textbf{Salas de computación}:
\begin{itemize}
\item Capacidad para al menos 14 computadores por sección (grupos de tres estudiantes, hasta 40 estudiantes por curso).
\item Conexión a Internet estable para acceder a las simulaciones alojadas en un servidor externo.
\end{itemize}
\item \textbf{Software}: navegador web actualizado; no se requiere instalación de software adicional en los equipos de los estudiantes dado que las simulaciones estarán sincronizadas en un Cloud de Computación.
\item \textbf{Materiales impresos}: guías de actividades y hojas de desarrollo por sesión; copias del cuestionario pre–post.
\item \textbf{Profesor del curso}: responsable de la conducción de la sesión, explicación de objetivos y discusión final.
\item Los estudiantes trabajarán en \textbf{grupos de tres} (o máximo cuatro, en caso que el número de alumnos no sea un múltiplo de tres).
\item Cada grupo tendrá asignado un computador durante la sesión.
\item Se espera que las sesiones Gk y Fk serean anunciadas como \textbf{actividades obligatorias} y parte integrante de la evaluación de Tareas/Laboratorios del curso.
\end{itemize}

\vspace{1cm}

%======================================================================
\chapter{Banco de Ítems para Mini–Evaluaciones} 
\label{Banco de Ítems}
%======================================================================

Este capítulo recoge los ítems propuestos para las mini–evaluaciones. Cada conjunto de ítems puede ser utilizado de manera flexible según necesidad de la sesión correspondiente.

%--------------------------------------------------
\section{Cinemática 1 — Frenada de emergencia}
%--------------------------------------------------

\begin{enumerate}
\item El auto A viaja a \(v_A = 30\,\text{m/s}\) y el auto B a \(v_B = 20\,\text{m/s}\). La distancia inicial es \(D = 40\,\text{m}\) y la aceleración de frenado es \(a = 3\,\text{m/s}^2\). Usando la condición de seguridad \((v_A - v_B)^2 < 2aD\), determina si habrá colisión.

\begin{solutionbox}
se calcula \(\Delta v = v_A - v_B = 10\,\text{m/s}\) y \((\Delta v)^2 = 100\). Luego se evalúa \(2 a D = 2 \cdot 3 \cdot 40 = 240\). Como \(100 < 240\), se cumple la condición de seguridad y no hay colisión.
\end{solutionbox}

\item Dos autos se mueven con \(v_A = 27\,\text{m/s}\) y \(v_B = 18\,\text{m/s}\), separados por \(D = 35\,\text{m}\). Encuentra la aceleración mínima \(a_{\min}\) que asegura que no haya colisión.

\begin{solutionbox}
se usa la frontera \((v_A - v_B)^2 = 2 a_{\min} D\). Con \(\Delta v = 9\,\text{m/s}\), se tiene \(81 = 2 a_{\min} \cdot 35\). De aquí, \(a_{\min} = 81 / 70\,\text{m/s}^2\).
\end{solutionbox}

\item Con \(v_A = 32\,\text{m/s}\), \(v_B = 24\,\text{m/s}\) y aceleración \(a = 4\,\text{m/s}^2\):  
(a) Calcula la distancia mínima \(D_{\min}\).  
(b) Evalúa si la situación es segura cuando \(D = 20\,\text{m}\).

\begin{solutionbox}
(a) se calcula \(\Delta v = 8\,\text{m/s}\) y se usa \((\Delta v)^2 = 2 a D_{\min}\), de modo que \(64 = 2 \cdot 4 \cdot D_{\min}\) y \(D_{\min} = 8\,\text{m}\). (b) se compara \(D = 20\,\text{m}\) con \(D_{\min}\): como \(20 > 8\), la situación es segura.
\end{solutionbox}

\item Recordando que en la simulación trabajaste en el plano \((D, \Delta v)\), donde la frontera está dada por \(\Delta v = \sqrt{2aD}\), indica si la región segura se encuentra por encima o por debajo de la curva. Justifica usando la desigualdad de seguridad.

\begin{solutionbox}
la frontera viene dada por la igualdad \((\Delta v)^2 = 2 a D\). La condición de seguridad es \((\Delta v)^2 < 2 a D\), es decir, \(\Delta v < \sqrt{2 a D}\). Por tanto, la región segura corresponde a los puntos que quedan \textbf{por debajo} de la curva \(\Delta v = \sqrt{2 a D}\) en el plano \((D, \Delta v)\).
\end{solutionbox}

\item Sabiendo que \((\Delta v_1, D_1)\) es seguro, analiza si el par \((2\Delta v_1, 2D_1)\) es necesariamente seguro, necesariamente peligroso o depende del caso.

\begin{solutionbox}
la condición de seguridad \((\Delta v)^2 < 2 a D\) aplicada a \((\Delta v_1, D_1)\) da \((\Delta v_1)^2 < 2 a D_1\). Para el par \((2\Delta v_1, 2 D_1)\) la desigualdad se transforma en \(4 (\Delta v_1)^2 < 4 a D_1\), equivalente a \((\Delta v_1)^2 < a D_1\), que es \textit{más exigente} que la condición original. Por tanto, no es cierto en general que \((2\Delta v_1, 2 D_1)\) sea seguro: depende de cuán lejos del borde estaba el punto original.
\end{solutionbox}

\end{enumerate}

%--------------------------------------------------
\section{Balística 1 — Rescate de la pelota}
%--------------------------------------------------

\begin{enumerate}
\item Para \(v_0 = 20\,\text{m/s}\), calcula \(\alpha = g / v_0^2\). Luego decide si el punto \(P_1 = (6\,\text{m}, 9\,\text{m})\) es alcanzable comparando con la parábola de seguridad.

\begin{solutionbox}
se calcula \(\alpha = g / v_0^2 \approx 9{,}8/400 = 0{,}0245\,\text{m}^{-1}\). La parábola de seguridad viene dada por \(y_{\text{seg}}(x) = \frac{1-(\alpha x)^2}{2 \alpha}\). Evaluando en \(x=6\), se obtiene \(y_{\text{seg}}(6) \approx 19{,}97\,\text{m}\). Como \(9 < 19{,}97\), el punto es alcanzable.
\end{solutionbox}

\item Para \(v_0 = 20\,\text{m/s}\):  
(a) Decide si el punto \(P_2 = (4\,\text{m}, 10\,\text{m})\) es alcanzable.  
(b) Repite para \(P_3 = (10\,\text{m}, 5\,\text{m})\).  
(c) Indica cuál está más cerca del límite de lo alcanzable.

\begin{solutionbox}
para cada punto \(P_2, P_3\) se evalúa \(y_{\text{seg}}(x)\) en la abscisa correspondiente y se compara la ordinada del punto con el valor de la parábola de seguridad. Un punto está “en el límite” cuando su altura es muy próxima a \(y_{\text{seg}}(x)\); si la altura es claramente menor, está bien dentro de la región alcanzable.
\end{solutionbox}

\item Escribe \(\alpha = g / v_0^2\). Si reemplazas \(v_0\) por \(2v_0\), expresa \(\alpha_{\text{nuevo}}\) en términos de \(\alpha\) y discute cómo cambia la región alcanzable.

\begin{solutionbox}
al reemplazar \(v_0\) por \(2 v_0\) se tiene \(\alpha_{\text{nuevo}} = g / (2 v_0)^2 = g / (4 v_0^2) = \alpha / 4\). Como la altura máxima de la parábola es proporcional a \(1/\alpha\), al disminuir \(\alpha\) a la cuarta parte, la región alcanzable se hace cuatro veces “más alta” y, en general, más amplia.
\end{solutionbox}

\item Explica por qué, cuando un punto \((x,y)\) es alcanzable, suelen existir dos soluciones de ángulo: un ángulo bajo y un ángulo alto.

\begin{solutionbox}
la ecuación de la trayectoria para un proyectil permite, para un mismo punto alcanzable, dos soluciones de ángulo \(\theta\): una con \(\theta\) pequeño (arco bajo) y otra con \(\theta\) grande (arco alto). Esto se ve al resolver en \(\theta\) la ecuación \(y = x \tan\theta - \tfrac{g}{2 v_0^2 \cos^2\theta} x^2\), que lleva a una ecuación cuadrática en \(\tan\theta\) con dos raíces reales cuando el punto está debajo de la parábola de seguridad.
\end{solutionbox}

\end{enumerate}

%--------------------------------------------------
\section{Balística 2 — Desafío del arquero}
%--------------------------------------------------

\begin{enumerate}
\item Usando  
\(y_f(x_c) = x_c \tan\theta - \frac{g}{2v_0^2 \cos^2\theta} x_c^2\),  
calcula la altura al pasar por \(x = x_c\).

\begin{solutionbox}
se reemplaza el valor de \(x_c\) en la expresión dada para obtener \(y_f(x_c)\) en función de \(v_0\) y \(\theta\); si se dan valores numéricos de \(v_0\) y \(\theta\), se calcula directamente la altura correspondiente en ese punto.
\end{solutionbox}

\item Dados dos anillos con centros \((x_{c1}, y_{c1})\), \((x_{c2}, y_{c2})\) y radios \(R_1, R_2\), evalúa para un disparo \((v_0,\theta)\) si pasa por ambos verificando:  
\(-R_i < y_f(x_{ci}) - y_{ci} < R_i\).

\begin{solutionbox}
primero se calculan \(y_f(x_{c1})\) y \(y_f(x_{c2})\) usando la fórmula del ítem anterior. Luego se verifican las dos inecuaciones \(-R_1 < y_f(x_{c1}) - y_{c1} < R_1\) y \(-R_2 < y_f(x_{c2}) - y_{c2} < R_2\). Si ambas se cumplen, el disparo pasa por los dos anillos; si una falla, no pasa por ese anillo.
\end{solutionbox}

\item Da dos modificaciones en los anillos que siempre reducen el espacio de soluciones \((\theta, v_0)\). Explica por qué.

\begin{solutionbox}
ejemplos típicos son: (i) disminuir el radio de uno o ambos anillos, lo que estrecha el intervalo \(y_c - R < y_f < y_c + R\) y reduce el conjunto de \((\theta, v_0)\) que lo satisfacen; (ii) aumentar la distancia entre los centros de los anillos, de modo que la trayectoria deba ajustar dos condiciones más separadas, lo que también reduce el espacio de soluciones.
\end{solutionbox}

\item El tiempo total de vuelo es \(T = \frac{2 v_0 \sin\theta}{g}\). Explica por qué la componente vertical \(v_0 \sin\theta\) es la que más afecta el tiempo de vuelo, y cuál de las dos soluciones (arco alto o bajo) tendrá mayor tiempo de vuelo.

\begin{solutionbox}
se observa que \(T\) depende linealmente de \(v_0 \sin\theta\), es decir, del componente vertical de la velocidad inicial. Para un mismo \(v_0\), el arco alto tiene \(\sin\theta\) mayor que el arco bajo, por lo que su tiempo de vuelo es mayor. En consecuencia, la trayectoria de arco alto es más “alta y lenta” que la de arco bajo.
\end{solutionbox}

\item Explica por qué la solución de arco alto tiende a facilitar el paso por dos anillos consecutivos, relacionándolo con el tiempo de vuelo y la altura máxima.

\begin{solutionbox}
el arco alto tiene un tiempo de vuelo mayor y alcanza alturas más elevadas, lo que facilita satisfacer simultáneamente dos condiciones de paso (dos anillos) ubicadas a distintas posiciones. La trayectoria permanece más tiempo en el aire, permitiendo ajustar mejor la altura en ambos puntos \(x_{c1}\) y \(x_{c2}\).
\end{solutionbox}

\item Considera el caso en que los dos anillos tienen el mismo centro y radio. Muestra cómo se reduce el sistema  
\(-R_i < y_f(x_{ci}) - y_{ci} < R_i\)  
y por qué no cambia la región de soluciones respecto al caso de un solo anillo.

\begin{solutionbox}
al tener \((x_{c1}, y_{c1}, R_1) = (x_{c2}, y_{c2}, R_2)\), las dos condiciones \(-R_1 < y_f(x_{c1}) - y_{c1} < R_1\) y \(-R_2 < y_f(x_{c2}) - y_{c2} < R_2\) se reducen a una sola inecuación. Por tanto, el conjunto de soluciones \((\theta, v_0)\) es exactamente el mismo que en el caso de un solo anillo en esa posición.
\end{solutionbox}

\end{enumerate}

%--------------------------------------------------
\section{Balística 3 — La bola en la escalera}
%--------------------------------------------------

\begin{enumerate}

\item Usando que el movimiento es horizontal con rapidez constante, escribe la ecuación \(x(t) = v_0 t\) y deduce el tiempo en que la bola alcanza el borde horizontal del escalón \(n\), ubicado en \(x_n = n w\).

\begin{solutionbox}
se usa \(x(t) = v_0 t\) y se iguala a \(x_n = n w\), de modo que \(v_0 t_n = n w\). Por tanto, el tiempo en que la bola llega al escalón \(n\) es \(t_n = x_n / v_0 = n w / v_0\).
\end{solutionbox}

\item Usando la ecuación vertical \(y(t) = -\tfrac12 g t^2\) y la expresión del tiempo obtenida en el ítem anterior, escribe la altura \(y_n\) de la bola al llegar a la posición \(x_n = n w\).

\begin{solutionbox}
se reemplaza \(t_n = n w / v_0\) en \(y(t) = -\tfrac12 g t^2\), obteniendo \(y_n = -\tfrac12 g (n w / v_0)^2\). Esta es la altura (negativa) de la bola al llegar a la posición \(x_n\).
\end{solutionbox}

\item Explica la condición que debe cumplir \(y_n\) para que la bola impacte el escalón \(n\) y no el escalón anterior. Es decir, expresa una condición del tipo “la bola llega entre la parte superior del escalón \(n\) y la parte superior del escalón \(n-1\)”.

\begin{solutionbox}
si los escalones tienen altura \(h\), la parte superior del escalón \(n\) está a \(y = -n h\) y la del escalón anterior a \(y = -(n-1)h\). La condición típica de impacto es que \(y_n\) quede entre ambos niveles, por ejemplo \(-n h \le y_n < -(n-1)h\), o una desigualdad equivalente según la convención usada en la guía.
\end{solutionbox}

\item Recordando que, en el desarrollo de la guía, se obtuvo que el número de escalón de impacto viene dado por
\[
  n = \left\lceil \frac{2 h v_0^2}{g w^2} \right\rceil,
\]
 discute qué ocurre con \(n\) si se reemplaza \(v_0\) por \(2 v_0\), manteniendo fijos \(h, g\) y \(w\).

\begin{solutionbox}
la expresión \(n = \left\lceil 2 h v_0^2 / (g w^2) \right\rceil\) muestra que \(n\) crece proporcionalmente a \(v_0^2\). Si se reemplaza \(v_0\) por \(2 v_0\), el factor \(v_0^2\) se multiplica por 4 y, en consecuencia, el valor dentro del techo se multiplica aproximadamente por 4. Esto significa que el escalón de impacto se desplaza a valores de \(n\) mucho mayores (aproximadamente cuatro veces más adelante, salvo ajustes por el techo).
\end{solutionbox}

\item Usando la expresión
\[
  n = \left\lceil \frac{2 h v_0^2}{g w^2} \right\rceil,
\]
discute el efecto de aumentar \(w\) o de aumentar \(h\) (manteniendo \(v_0\) fijo) sobre el número de escalón de impacto. ¿Qué cambio produce cada modificación sobre \(n\)?

\begin{solutionbox}
en \(n = \left\lceil 2 h v_0^2 / (g w^2) \right\rceil\), si \(w\) aumenta, el denominador \(w^2\) crece y el cociente disminuye, por lo que \(n\) tiende a disminuir (la bola golpea un escalón más cercano). Si \(h\) aumenta, el numerador crece y el cociente aumenta, por lo que \(n\) tiende a aumentar (la bola impacta un escalón más lejano).
\end{solutionbox}

\end{enumerate}

%--------------------------------------------------
%--------------------------------------------------
\section{Sólido de Revolución 1 — Cono}
%--------------------------------------------------

\begin{enumerate}
\item Para un cono recto cuyo eje es vertical y cuyo ángulo con la vertical es \(\theta\), muestra que la distancia radial al eje a una altura \(h\) cumple
\[
  r(h) = h \tan\theta.
\]

\begin{solutionbox}
al considerar el triángulo rectángulo formado por la altura \(h\) y el radio \(r\), se tiene \(\tan\theta = r/h\). Despejando, se obtiene \(r(h) = h \tan\theta\).
\end{solutionbox}

\item Considera una masa que describe un movimiento circular uniforme de radio \(r\) en el interior del cono. A partir de las proyecciones de fuerzas (peso y normal), muestra que la condición de giro estable viene dada por
\[
  \frac{v^2}{r} = g \tan\theta.
\]

\begin{solutionbox}
se descompone la fuerza normal \(N\) en componentes: \(N \cos\theta = mg\) (equilibrio vertical) y \(N \sin\theta = m v^2 / r\) (centrípeta). Dividiendo la segunda por la primera se obtiene \(\tan\theta = v^2 / (r g)\), es decir, \(v^2 / r = g \tan\theta\).
\end{solutionbox}

\item Usando las expresiones anteriores, reemplaza \(r(h) = h \tan\theta\) en la condición de giro estable y obtiene una expresión explícita para \(v(h)\).

\begin{solutionbox}
sustituyendo \(r = h \tan\theta\) en \(v^2 / r = g \tan\theta\) se obtiene \(v^2 = g r \tan\theta = g (h \tan\theta) \tan\theta = g h \tan^2\theta\). Por tanto, \(v(h) = \tan\theta\, \sqrt{g h}.\)
\end{solutionbox}

\item Compara las velocidades de giro \(v(h_1)\) y \(v(h_2)\) para dos alturas tales que \(0 < h_1 < h_2\). ¿Qué implicancia tiene esto sobre cómo cambia la velocidad requerida para un giro estable al aumentar la altura en el cono?

\begin{solutionbox}
como \(v(h) = \tan\theta \sqrt{g h}\), para un mismo \(\theta\) la velocidad crece con \(\sqrt{h}\). Si \(0 < h_1 < h_2\), entonces \(v(h_1) < v(h_2)\): a mayor altura en el cono, mayor velocidad de giro requerida.
\end{solutionbox}

\item Considera ahora dos conos con ángulos \(\theta_1 < \theta_2\). Para una misma altura \(h\), usa la expresión de \(v(h)\) para decidir en cuál de los dos conos se requiere una mayor velocidad de giro para mantener la órbita estable.

\begin{solutionbox}
para una altura fija \(h\), se tiene \(v(h) = \tan\theta \sqrt{g h}\). Si \(\theta_1 < \theta_2\), entonces \(\tan\theta_1 < \tan\theta_2\), por lo que \(v_1(h) < v_2(h)\). El cono más abierto (mayor \(\theta\)) requiere una velocidad de giro mayor para mantener la órbita estable.
\end{solutionbox}

\item Analiza el límite de la expresión \(v(h) = \tan\theta \, \sqrt{g h}\) cuando \(\theta \to 0\). ¿Qué ocurre con la velocidad de giro estable en ese límite y qué interpretación física tiene este resultado?

\begin{solutionbox}
en \(v(h) = \tan\theta \, \sqrt{g h}\), al tomar el límite \(\theta \to 0\) se tiene \(\tan\theta \to 0\) y, por tanto, \(v(h) \to 0\). Esto indica que, en el límite de un cono muy angosto (\(\theta\) casi cero), no es posible mantener un giro estable con velocidad finita: la partícula tendería a deslizarse hacia el vértice.
\end{solutionbox}

\end{enumerate}

%--------------------------------------------------
\section{Sólido de Revolución 2 — Superficies generales}
%--------------------------------------------------

\begin{enumerate}
\item Explica por qué, para una superficie de revolución descrita por \(r(h)\), el ángulo local \(\varphi\) entre la superficie y la horizontal cumple
\[
  \tan\varphi = \left|\frac{dr}{dh}\right|.
\]
Interpreta geométricamente esta relación en términos de la pendiente de la curva meridiana \(r(h)\).

\begin{solutionbox}
en el corte meridiano, la superficie se describe por la curva \(r(h)\). La pendiente de esta curva respecto de la horizontal es \(dr/dh\), que geométricamente corresponde a \(\tan\varphi\). El valor absoluto aparece porque sólo interesa el ángulo con la horizontal, no si la pendiente es positiva o negativa.
\end{solutionbox}

\item Usando las expresiones generales
\[
  \frac{v^2}{r} = g \tan\varphi, \qquad \tan\varphi = \left|\frac{dr}{dh}\right|,
\]
 muestra que la velocidad de giro estable satisface, en general,
\[
  v^2(h) = g\, r(h) \, \left|\frac{dr}{dh}\right|.
\]

\begin{solutionbox}
a partir de \(v^2 / r = g \tan\varphi\) y \(\tan\varphi = |dr/dh|\), se sustituye en la primera ecuación para obtener \(v^2 / r = g |dr/dh|\). Multiplicando ambos lados por \(r\) se llega a \(v^2(h) = g\, r(h)\, |dr/dh|\).
\end{solutionbox}

\item Considera ahora dos superficies de revolución:
\begin{itemize}
  \item Superficie A (cono): \(r(h) = k h\);
  \item Superficie B (cuadrática simple): \(r(h) = k h^2\),
\end{itemize}
con \(k>0\). Calcula \(v(h)\) en cada caso usando la expresión general del ítem anterior y compara cómo crece la velocidad de giro requerida con la altura en ambas superficies.

\begin{solutionbox}
para el cono \(r(h) = k h\), se tiene \(dr/dh = k\) y, por tanto, \(v^2(h) = g (k h) k = g k^2 h\), de donde \(v(h) \propto h^{1/2}\). Para la superficie cuadrática \(r(h) = k h^2\), se tiene \(dr/dh = 2 k h\) y \(v^2(h) = g (k h^2) (2 k h) = 2 g k^2 h^3\), de modo que \(v(h) \propto h^{3/2}\). Así, en la superficie cuadrática la velocidad crece mucho más rápido con la altura que en el cono.
\end{solutionbox}

\item Discute, en términos de la fórmula general \(v^2(h) = g\, r(h)\, \left|dr/dh\right|\), por qué en superficies donde \(r(h)\) y \(\left|dr/dh\right|\) crecen rápidamente con la altura, la velocidad requerida para el giro estable aumenta de manera marcada al subir en la superficie.

\begin{solutionbox}
la expresión \(v^2(h) = g\, r(h)\, |dr/dh|\) muestra que, si tanto \(r(h)\) como \(|dr/dh|\) aumentan fuertemente con \(h\), su producto crece y, en consecuencia, también lo hace \(v^2(h)\). Esto implica que la velocidad requerida para mantener el giro estable se vuelve mucho mayor al subir en la superficie.
\end{solutionbox}

\item Considera la superficie de revolución paraboloidal definida por
\[
  r(h) = k \sqrt{h},
\]
con \(k>0\). Calcula \(\frac{dr}{dh}\) y luego usa la fórmula general \(v^2(h) = g\, r(h)\, \left|dr/dh\right|\) para mostrar que la velocidad de giro estable \(v(h)\) no depende de la altura. Explica por qué, en este sentido, esta superficie puede considerarse aproximadamente isócrona.

\begin{solutionbox}
para \(r(h) = k \sqrt{h}\) se tiene \(dr/dh = k / (2 \sqrt{h})\). Entonces \(v^2(h) = g (k \sqrt{h}) (k / (2 \sqrt{h})) = g k^2/2\), que es constante. Por tanto, \(v(h)\) no depende de \(h\) y la superficie se comporta como isócrona: a distintas alturas se requiere prácticamente la misma velocidad de giro.
\end{solutionbox}

\end{enumerate}

% End of inserted chapter

\end{document}